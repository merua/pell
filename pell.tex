\documentclass[a4paper,12pt]{article}

\usepackage[utf8]{inputenc}
\usepackage[ngerman]{babel}
\usepackage{amsmath}
\usepackage{amssymb}
\usepackage{amsfonts}
\usepackage{geometry}
\geometry{a4paper, top=2cm, bottom=2cm, left=2cm, right=2cm}
\pagestyle{empty}

\newcounter{task}
\renewcommand{\thesubsection}{\arabic{task}}
\newcommand{\task}{\stepcounter{task}\subsection}

\begin{document}
\section*{Übungsblatt Pellsche Gleichungen}

Im Folgenden sollen Lösungen immer ganzzahlige Lösungen sein.
\task{Aufwärmen}

Finde durch geschicktes Ausprobieren eine Lösung von \[x^2-5y^2=1\text{.}\] Kannst Du mit dem Satz von Lagrange alle Lösungen beschreiben?

\task{Wieso sind die Voraussetzungen für normale Pellsche Gleichungen sinnvoll?}

\begin{enumerate}
    \item Welche Lösungen hat $x^2-dy^2=1$, falls $d$ eine Quadratzahl ist?
    \item Wieso hat $x^2-dy^2=n$ für $d<0$ höchstens endlich viele Lösungen? Kannst Du eine Abschätzung für die Anzahl Lösungen angeben?
    \item Wieso hat $x^2-3y^2=2$ keine Lösungen?
\end{enumerate}

\task{Wenn man den Witz erklärt, ist er nicht mehr lustig.}
Im Folgenden soll die Lösungsstruktur der \emph{verallgemeinerten} Pellschen Gleichung

\begin{align}\label{verallg}
x^2-dy^2=n    
\end{align}

für $n\neq 0$ beschrieben werden.

Sei $u=a+b\sqrt d>1$ die Fundamentallösung von $x^2-dy^2=1$. Zeige, dass es für jede Lösung der Lösung $(x,y)$ von (\ref{verallg}) eine weitere Lösung $(x',y')$ der Form \[
(x'+y'\sqrt{d})=(x+y\sqrt{d})u^k
\] gibt, welche \[\vert x'\vert\leq\frac{u+\vert n\vert}{2}\] und \[\vert y'\vert\leq\frac{u+\vert n\vert}{2\sqrt{d}}\] erfüllt.

\begin{enumerate}
    \item Interpretiere das Ergebnis.
    \item Die Gleichung $x^2-37y^2=1$ hat die Fundamentallösung $u=73+12\sqrt{37}$. Zeige (evtl. unter Zuhilfenahme eines Taschenrechners), dass $x^2-37y^2=11$ gar keine Lösungen hat.
\end{enumerate}
\end{document}
